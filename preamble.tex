% Imports
\usepackage{amsmath,amssymb,amsthm,answers,booktabs,color,esvect,fancyhdr,
    framed,listings,mathtools,multicol,pbox,pdfpages,tabu,tikz,tikz-3dplot,
    tkz-euclide,varwidth}
\usepackage[hidelinks]{hyperref}
\usepackage{cleveref}[2012/02/15]
\crefformat{footnote}{#2\footnotemark[#1]#3}
\usepackage[inline]{enumitem}
\usepackage[squaren,Gray]{SIunits}
\usepackage[many]{tcolorbox}
\usepackage{biblatex}
\addbibresource{cite.bib}

% Color scheme
\definecolor{accent}{rgb}{0,0,1}
\definecolor{accentBack}{rgb}{.75,.75,1}
\definecolor{pos}{rgb}{0,0,1}
\definecolor{posBack}{rgb}{.75,.75,1}
\newcommand{\posName}{blue}
\definecolor{neg}{rgb}{1,0,0}
\definecolor{negBack}{rgb}{1,.75,.75}
\newcommand{\negname}{red}
\definecolor{colorOK}{rgb}{0,.4,0}
\definecolor{colorKO}{rgb}{.8,0,0}
\definecolor{lg}{rgb}{.8,.8,.8}

% Utilities
\newcommand{\todo}[1]{\textbf{TODO:} #1}
\newcommand{\term}[1]{\emph{#1}}
\newcommand{\textOK}[1]{\textcolor{colorOK}{#1}}
\newcommand{\textKO}[1]{\textcolor{colorKO}{#1}}
\newcommand{\OK}{\textOK{OK}}
\newcommand{\KO}{\textKO{KO}}
\newcommand{\spoiler}[1]{{\color{gray}(select to reveal)} {\color{white} #1}}
\newcommand{\ttt}[1]{\texttt{#1}}
\newcommand{\includeFig}[1]{\includegraphics{figs-out/#1}}
\newcommand{\centerFig}[1]{
    \begin{center}
    \includeFig{#1}
    \end{center}
}

% Math operators
\DeclarePairedDelimiter{\floor}{\lfloor}{\rfloor}
\DeclarePairedDelimiter{\ceil}{\lceil}{\rceil}
\DeclarePairedDelimiter{\norm}{\|}{\|}
\DeclarePairedDelimiter{\abs}{|}{|}
\DeclareMathOperator{\orient}{orient}
\DeclareMathOperator{\acos}{acos}
\DeclareMathOperator{\atanTwo}{atan2}
\DeclareMathOperator{\cis}{cis}
\DeclareMathOperator{\perpop}{perp}
\DeclareMathOperator{\side}{side}
\DeclareMathOperator{\round}{round}
\DeclareMathOperator{\dist}{dist}
\newcommand{\real}{\mathbb{R}}
\newcommand{\complex}{\mathbb{C}}
\newcommand{\turn}{2\pi}
\newcommand{\half}{\pi}
\newcommand{\quarter}{\frac{\pi}{2}}
\newcommand{\conj}[1]{\overline{#1}}
\newcommand{\proj}[2]{\pi_{#1}(#2)}
\newcommand{\projv}[2]{\proj{\vv{#1}}{\vv{#2}}}
\newcommand{\normv}[1]{\norm{\vv{#1}}}
\newcommand{\dotv}[2]{\vv{#1}\cdot\vv{#2}}
\newcommand{\crossv}[2]{\vv{#1}\times\vv{#2}}
\newcommand{\dir}{\vv{v}}
\newcommand{\dirof}[1]{\vv{v_{#1}}}
\newcommand{\intbis}{l_\mathrm{int}}
\newcommand{\extbis}{l_\mathrm{ext}}
\renewcommand{\phi}{\varphi}
\newcommand{\nth}{^{\mathrm{th}}}
\newcommand{\eMach}{\epsilon_{\mathrm{machine}}}
\newcommand{\lint}{{\intbis(l_1,l_2)}}
\newcommand{\lext}{{\extbis(l_1,l_2)}}


% Listings
\usepackage[scaled=0.8]{DejaVuSansMono}
\lstset{
    language=C++,
    basicstyle=\ttfamily,
    commentstyle=\it\ttfamily\color{gray},
    keywordstyle=\bf\ttfamily,
    stringstyle=\color{gray},
    showstringspaces=false,
    breaklines=true,
    morekeywords={decltype},
    frame=single
}

% Spacing
\setlength{\parskip}{.4em plus.1em minus.1mm}
\setlist{topsep=-0.1em,itemsep=-.2em}
\tabulinesep=.1cm

% Box environments
\newtcolorbox{mybox}[1]{
    tikznode boxed title,
    enhanced,
    arc=2mm,
    interior style={white},
    attach boxed title to top center= {yshift=-\tcboxedtitleheight/2},
    fonttitle=\bfseries,
    colbacktitle=white,coltitle=black,
    boxed title style={size=normal,colframe=white,boxrule=0pt},
    title={#1}}
\newenvironment{mathy}{\begin{mybox}{Math insight}}{\end{mybox}}
\newenvironment{warning}{\begin{mybox}{Warning}}{\end{mybox}}
\newenvironment{trick}{\begin{mybox}{Implementation trick}}{\end{mybox}}
\newenvironment{myc}{\begin{varwidth}{\textwidth}\centering}{\end{varwidth}}
\newcommand{\vc}[2]{
    \begin{myc}
    #1
    
    #2
    \end{myc}
}

% Exercises
\Newassociation{sol}{Solution}{solutions}
\newcounter{ExerciseCounter}
\newenvironment{exerciseMain}{
    \begin{mybox}{Exercise \arabic{ExerciseCounter}}
}{
    \end{mybox}
}
\newenvironment{exerciseAppendix}[1]{
    \hypertarget{sol:#1}{}
    \begin{mybox}{Exercise \ref{ex:#1}}
}{
    \end{mybox}
}
\newcommand{\exo}[1]{
    \stepcounter{ExerciseCounter}
    \begin{exerciseMain}
    #1
    \end{exerciseMain}
}

\newcommand{\exoWithSolution}[3]{
    \refstepcounter{ExerciseCounter}\label{ex:#3}
    \begin{exerciseMain}
        #1
        
        \flushright {\color{gray}\hyperlink{sol:#3}{[Go to solution]}}
    \end{exerciseMain}
    
    \Writetofile{solutions}{
        \protect \begin{exerciseAppendix}{#3}
            #1
        \protect \end{exerciseAppendix}
        #2
        
        \protect\begin{flushright}
            \protect\color{gray} \protect\hyperref[ex:#3]{[Back to exercise]}
        \protect\end{flushright}
    }
}

% Theorems and definitions
\newtheorem{assum}{Assumption}
\newtheorem{mydef}{Definition}
\newtheorem{thm}{Theorem}
\newtheorem{lemma}{Lemma}
